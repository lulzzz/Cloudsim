\documentclass[12pt]{article}
\usepackage[margin=1in]{geometry}
\usepackage{indentfirst}
\usepackage [english]{babel}
\usepackage [autostyle, english = american]{csquotes}
\MakeOuterQuote{"}
\usepackage{graphicx}
\usepackage{amsmath}
\usepackage{tikz}


\begin{document}


\title{\textbf{Cloud Computing Simulation for Smart Grid}}
\author{Hanlin Chen\\\\the Department of Electrical and Computer Engineering\\Ohio State University\\ chen.8059@buckeyemail.osu.edu}
\maketitle 

\section{Introduction}
The growth of power  data in smart gird motivates use of high performance cloud computing  application. In the cloud model, all data processing, software and hardware are delivered to users via internet in forms of service such as platform as a service (Paas), software as a service (Saas) and infrastructure as a service (Iaas). Compared to traditional computing framework, the cloud computing can handle much higher volume of data from smart grid and provide promising power management to users. The performance of cloud model could be addressed by its processing speed and throughput. In addition, various factors such as hardware infrastructures, allocation of virtual environment and smart grid application are directly linked to these performance measurements. In this paper, we proposed a novel cloud based data center architectures to improve data processing in the smart power grid. 
The runtime simulation is done to investigate the effect of hardware configuration and allocation policy of virtual machines on execution of smart grid application using open source toolkit Cloudsim \cite{ok}. The structure of this paper is as follow. Section II reviews overall architectures of proposed cloud based data center for smart grid. Section III studies two common VM scheduling: time shared policy and space shared policy. Section IV includes all simulation results and time analysis. Section V gives the conclusion. 

\section{Cloud-Based Data Center Architectures}

The basic cloud architectures enable supports for maximizing the availability and increasing demand of computing power and data storage\cite{ok2}. In our experiment, the cloud computing service is provide by one power company. Therefore, only 1 Cloud Service Provider (CSP) or 1 data center is required in our simulation. The data center is accountable for providing seamless infrastructure services for all cloud users in the city. In order to achieved higher utilization of computing, the virtualization of hardware is used and allocating computing hardware on user's request. The advantage of virtualization is that the service provider is not obligated to allocate all  hardware resources in advance and charge users based on usage of storage and bandwidth. In addition, it can also prevent applications from interrupting each other during execution, and provide higher data security. the detail of cloud-based data center architectures are displayed in figure 1.
\newpage
\bibliographystyle{IEEEtran}
\bibliography{ref}



\end{document}